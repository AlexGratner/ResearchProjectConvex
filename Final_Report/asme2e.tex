%%% use twocolumn and 10pt options with the asme2e format
\documentclass[twocolumn,10pt]{asme2e}
\special{papersize=8.5in,11in}

%% The class has several options
%  onecolumn/twocolumn - format for one or two columns per page
%  10pt/11pt/12pt - use 10, 11, or 12 point font
%  oneside/twoside - format for oneside/twosided printing
%  final/draft - format for final/draft copy
%  cleanfoot - take out copyright info in footer leave page number
%  cleanhead - take out the conference banner on the title page
%  titlepage/notitlepage - put in titlepage or leave out titlepage
%  
%% The default is oneside, onecolumn, 10pt, final

%%% Replace here with information related to your conference
\confshortname{IDETC/CIE 2009}
\conffullname{the ASME 2009 International Design Engineering Technical Conferences \&\\
              Computers and Information in Engineering Conference}

%%%%% for date in a single month, use
%\confdate{24-28}
%\confmonth{September}
%%%%% for date across two months, use
\confdate{August 30-September 2}
\confyear{2009}
\confcity{San Diego}
\confcountry{USA}

%%% Replace DETC2009/MESA-12345 with the number supplied to you 
%%% by ASME for your paper.
\papernum{DETC2009/MESA-12345}

%%% You need to remove 'DRAFT: ' in the title for the final submitted version.
\title{DRAFT: AN ARTICLE CREATED USING \LaTeX2\raisebox{-.3ex}{$\epsilon$}\ IN ASME FORMAT}

%%% first author
\author{Yuchao Li
    \affiliation{
	Integration Engineering Laboratory\\
	Department of Mechanical and Aeronautical Engineering\\
	University of California\\
	Stockholm, Sweden\\
    Email: yuchao@kth.se
    }	
}


\author{Anqing Duan%\thanks{Address all correspondence to this author.} \\   
    \affiliation{Department or Division Name\\
	Company or College Name\\
	City, State (spelled out), Zip Code\\
	Sweden \\
	Gratner@live.se
    }
}

\author{Alexander Gratner%\thanks{Address all correspondence to this author.} \\   
    \affiliation{Department or Division Name\\
	Company or College Name\\
	City, State (spelled out), Zip Code\\
	Sweden \\
	Gratner@live.se
    }
}

\begin{document}

\maketitle    

%%%%%%%%%%%%%%%%%%%%%%%%%%%%%%%%%%%%%%%%%%%%%%%%%%%%%%%%%%%%%%%%%%%%%%
\begin{abstract}
{\it This article illustrates preparation of ASME paper using \LaTeX2\raisebox{-.3ex}{$\epsilon$}. An abstract for an ASME paper should be less than 150 words and is normally in italics.}
\end{abstract}

%%%%%%%%%%%%%%%%%%%%%%%%%%%%%%%%%%%%%%%%%%%%%%%%%%%%%%%%%%%%%%%%%%%%%%
\begin{nomenclature}
\entry{A}{You may include nomenclature here.}
\entry{$\alpha$}{There are two arguments for each entry of the nomemclature environment, the symbol and the definition.}
\end{nomenclature}

The spacing between abstract and the text heading is two line spaces.  The primary text heading is  boldface in all capitals, flushed left with the left margin.  The spacing between the  text and the heading is also two line spaces.

%%%%%%%%%%%%%%%%%%%%%%%%%%%%%%%%%%%%%%%%%%%%%%%%%%%%%%%%%%%%%%%%%%%%%%
\section*{INTRODUCTION}
%\subsection*{Second-Level Heading}
%\subsubsection*{Third-Level Heading.}
%Paper number: ASME assigns each accepted paper with a unique number. Replace {\bf DETC98/DAC-1234} in the input file preamble (the location will be obvious) with the paper number supplied to you  by ASME for your paper.
%All figures should be positioned at the top of the page where possible.  All figures should be numbered consecutively and captioned; the caption uses all capital letters, and centered under the figure as shown in Fig.~\ref{figure_ASME}. All text within the figure should be no smaller than 7~pt. There should be a minimum two line spaces between figures and text. The number of a referenced figure or table in the text should be preceded by Fig.\ or Tab.\ respectively unless the reference starts a sentence in which case Fig.\ or Tab.\ should be expanded to Figure or Table.

% Designing mechatronic systems is a problematic task.. 
% These persons has therefore made significant contributions
% Bond Graph approach
% Genetic algorith approach
% Some other methodologies
% Accordning to this book, convex is better..
% Introducing convex optimization would go a long way in solving the design dilemma..



Could the current resaerch trend of using genetric algorithms as a solver of mechatronic design optimization problems be improved by adapting a diciplinary convex optimization approach? The multidiciplinary nature of mechatronic systems, which could be characterised as systems with synergetic integrations of mechanical engineering with eletronics and intelligent computer control \cite{harashima1999}, makes early design optimization a difficult process. Optimizations are often done in the latter detailed design phases \cite{(I CEDEX) Engineering design; A systematic approach, 3rd ed} whereas early design decisions becomes a limiting factor. Research on the topic of optimization in early design stages for mechatronic systems has presented both holistic and non-holistic approaches which share the common feature of using genetic algorithms, either as a complement or as a standalone solver. The disadvantages of GA when it comes to accuracy and computation time justifies research on applying diciplinary convex optimization.

\par
%Modern research within the field of mehcatronic systme optimization has...
Optimizing mechatronic systems using genetic algorithms has been widely explored by researchers today. Hammadi \cite{Hammadi} proposes an emergent multi-agent approach where the system is decoposed into smal sub-systems (or agents) that in turn is optimized using the genetic algorithms NSGA II. Another approach that uses NSGA II is proposed by Guizani \cite{GUizani} where a partioning method is utilized to decompose the mechatronic system and classification of the interactions between partions are done. An approach developed by Seo \cite{Seo} and further developed by Behbahani \cite{Behabani} uses a two loop optimization process which incorperates genetic algorithms and bond graphs in an outer loop to otimize system topology and genetic programming in an inner loop to fin the elite solution within the optimal topology. The research in this paper will be based upon the work done by Malmquiest et al. \cite{Malmquist} where they created a framework named IDIOM which optimizises mechatronic systems in a holistic manner by the use of genetic algorithms.

\par

The IDIOM framework developed by Malmquist et al. extends on the Roos \cite{Roos} on a research on a methodology for integrated design och mechatronic servo systems. In his research, Roos reflects on the choice of using genetic algorithms to optimize the servo systems by stating that "The drawback of this method for system optimization is that it is computationally intensive and that there exists no mathematical proof that it actually finds the true optimum". This statement justifies the exploration of other optimization techniques which could provide true optimizations in a less computationally intensive manner, and for which convex optimization has been chosen. Diciplinary convex programming is a methodology developed by Grant et al. \cite{Grant} which purpose, according to the authors, is to "allow much of the manipulation and transformation required to analyze and solve convex programs to be automated". The methodology originates from procedures taken by those who regularly study convex optmization problems and has been implemented in the modelling framework called CVX \cite{CVX}.  

\par
The argument made by Roos serve as a starting point in the research presented in this report. It challenges the current trend of using evolutionary algorithms when optimizing the design of mechatronic systems and could provide a optimization approach that is less compututionally expensive and more accurate. 

\par
The remainder of this paper will introduce the prevuous work done by Malmquist et al. of which this study aim to extend on. \$3 describes the method used when applying convex optimization to the mechatronic system. Finally, \$4 introduces a case study where a system optimization is performed, and compared to the results obtained by Malmquist et al.  , of a mechatronic system composed of a motor, planetary gear and a load.  


%Mechatronic systems can be characterised as systems that emphasize on the synergetic integration of mechanical engineering with electronics and intelligent computer control \cite{harashima1999}. This multidiciplinary nature of the systems makes early design optimization difficult which is why most optimization processes today are done in the latter detailed design phase \cite{Cedex2014}. However, these optimization processes become limited by early design decisions and are therefore not resulting in a truly optimal system. Previous research on the topic of early design optimization in mechatronic systems has presented various approaches whereas the majority utilizes evolutionary algorithms (such as genetic algorithms), either as complement or as a standalone optimization technique. Disadvantages when optimizing with GAs is that the method does not necessary find the optimal point and that response time become large when the parameter space increases \cite{Boyd}. These problems justifies a different optimization method, such as convex optimization, to be applied. 

%A disadvantage when optimizing with GAs is that the approach does not necessary produce the most optimal solution and the that the computational time increases a lot when applying it on complex systems.    

%Previous research has presented methodologies to solve this early design optimization. 

%A lot of methodologies has been presented within the field 

%An aspect shared by the majority of them


%The majority of approaches..


%optimizations done in the latter design stages are limited by early design decisions and are therefore not generating an truly optimal system.  
%a
%and this process is often done in later, detailed design phases \cite and the optimization is therefore constrained by design choices in early stages. Approaches or methodologies to solve this problem in early design phases. 

%FUNKAR DET?

%This section should include:
%\begin{enumerate}
%\item Problem statement
%	\subitem Emphasize (Justify that the problem is important)
%\item Analyze the gap (specific, technical)
%\item What will you do to fill the gap.
%\item Summarize contributions (Summarize step 1-3, emphasize on 1.2)
%\item Outlook of structure. 
%\end{enumerate}


\section*{MATHEMATICS}


%%%%%%%%%%%%%%%% begin equation %%%%%%%%%%%%%%%%%%% 
\subsection*{Sample equation}
Equation ~\ref{eq_ASME}
\begin{equation}
f(t) = \int_{0_+}^t F(t) dt + \frac{d g(t)}{d t}
\label{eq_ASME}
\end{equation}
%%%%%%%%%%%%%%%% end equation %%%%%%%%%%%%%%%%%%% 
%%%%%%%%%%%%%%%% begin figure %%%%%%%%%%%%%%%%%%% 
\subsection*{Sample Figure}
Figure ~\ref{figure_ASME}

\begin{figure}[t]
\begin{center}
\setlength{\unitlength}{0.012500in}%
\begin{picture}(115,35)(255,545)
\thicklines
\put(255,545){\framebox(115,35){}}
\put(275,560){Beautiful Figure}
\end{picture}
\end{center}
\caption{THE FIGURE CAPTION USES CAPITAL LETTERS.}
\label{figure_ASME} 
\end{figure}
%%%%%%%%%%%%%%%% end figure %%%%%%%%%%%%%%%%%%% 
%%%%%%%%%%%%%%% begin table   %%%%%%%%%%%%%%%%%%%%%%%%%%
\subsection*{Sample Table}
Figure ~\ref{figure_ASME}

\begin{table}[t]
\caption{THE TABLE CAPTION USES CAPITAL LETTERS, TOO.}
\begin{center}
\label{table_ASME}
\begin{tabular}{c l l}
& & \\ % put some space after the caption
\hline
Example & Time & Cost \\
\hline
1 & 12.5 & \$1,000 \\
2 & 24 & \$2,000 \\
\hline
\end{tabular}
\end{center}
\end{table}
%%%%%%%%%%%%%%%% end table %%%%%%%%%%%%%%%%%%% 

All tables should be numbered consecutively and  captioned; the caption should use all capital letters, and centered above the table as shown in Table~\ref{table_ASME}. The body of the table should be no smaller than 7 pt.  There should be a minimum two line spaces between tables and text.

%%%%%%%%%%%%%%%%%%%%%%%%%%%%%%%%%%%%%%%%%%%%%%%%%%%%%%%%%%%%%%%%%%%%%%
\section*{FOOTNOTES\protect\footnotemark}
\footnotetext{Examine the input file, asme2e.tex, to see how a footnote is given in a head.}

Footnotes are referenced with superscript numerals and are numbered consecutively from 1 to the end of the paper\footnote{Avoid footnotes if at all possible.}. Footnotes should appear at the bottom of the column in which they are referenced.


%%%%%%%%%%%%%%%%%%%%%%%%%%%%%%%%%%%%%%%%%%%%%%%%%%%%%%%%%%%%%%%%%%%%%%
\section*{CITING REFERENCES}

%%%%%%%%%%%%%%%%%%%%%%%%%%%%%%%%%%%%%%%%%%%%%%%%%%%%%%%%%%%%%%%%%%%%%%
The ASME reference format is defined in the authors kit provided by the ASME.  The format is:

\begin{quotation}
{\em Text Citation}. Within the text, references should be cited in  numerical order according to their order of appearance.  The numbered reference citation should be enclosed in brackets.
\end{quotation}

The references must appear in the paper in the order that they were cited.  In addition, multiple citations (3 or more in the same brackets) must appear as a `` [1-3]''.  A complete definition of the ASME reference format can be found in the  ASME manual \cite{asmemanual}.

The bibliography style required by the ASME is unsorted with entries appearing in the order in which the citations appear. If that were the only specification, the standard {\sc Bib}\TeX\ unsrt bibliography style could be used. Unfortunately, the bibliography style required by the ASME has additional requirements (last name followed by first name, periodical volume in boldface, periodical number inside parentheses, etc.) that are not part of the unsrt style. Therefore, to get ASME bibliography formatting, you must use the \verb+asmems4.bst+ bibliography style file with {\sc Bib}\TeX. This file is not part of the standard BibTeX distribution so you'll need to place the file someplace where LaTeX can find it (one possibility is in the same location as the file being typeset).

With \LaTeX/{\sc Bib}\TeX, \LaTeX\ uses the citation format set by the class file and writes the citation information into the .aux file associated with the \LaTeX\ source. {\sc Bib}\TeX\ reads the .aux file and matches the citations to the entries in the bibliographic data base file specified in the \LaTeX\ source file by the \verb+\bibliography+ command. {\sc Bib}\TeX\ then writes the bibliography in accordance with the rules in the bibliography .bst style file to a .bbl file which \LaTeX\ merges with the source text.  A good description of the use of {\sc Bib}\TeX\ can be found in \cite{latex, goosens} (see how 2 references are handled?).  The following is an example of how three or more references \cite{latex, asmemanual,  goosens} show up using the \verb+asmems4.bst+ bibliography style file in conjunction with the \verb+asme2e.cls+ class file. Here are some more \cite{art, blt, ibk, icn, ips, mts, mis, pro, pts, trt, upd} which can be used to describe almost any sort of reference.

% Here's where you specify the bibliography style file.
% The full file name for the bibliography style file 
% used for an ASME paper is asmems4.bst.
\bibliographystyle{asmems4}


%%%%%%%%%%%%%%%%%%%%%%%%%%%%%%%%%%%%%%%%%%%%%%%%%%%%%%%%%%%%%%%%%%%%%%
\begin{acknowledgment}
Thanks go to D. E. Knuth and L. Lamport for developing the wonderful word processing software packages \TeX\ and \LaTeX. I also would like to thank Ken Sprott, Kirk van Katwyk, and Matt Campbell for fixing bugs in the ASME style file \verb+asme2e.cls+, and Geoff Shiflett for creating 
ASME bibliography stype file \verb+asmems4.bst+.
\end{acknowledgment}

%%%%%%%%%%%%%%%%%%%%%%%%%%%%%%%%%%%%%%%%%%%%%%%%%%%%%%%%%%%%%%%%%%%%%%
% The bibliography is stored in an external database file
% in the BibTeX format (file_name.bib).  The bibliography is
% created by the following command and it will appear in this
% position in the document. You may, of course, create your
% own bibliography by using thebibliography environment as in
%
% \begin{thebibliography}{12}
% ...
% \bibitem{itemreference} D. E. Knudsen.
% {\em 1966 World Bnus Almanac.}
% {Permafrost Press, Novosibirsk.}
% ...
% \end{thebibliography}

% Here's where you specify the bibliography database file.
% The full file name of the bibliography database for this
% article is asme2e.bib. The name for your database is up
% to you.
\bibliography{asme2e}

%%%%%%%%%%%%%%%%%%%%%%%%%%%%%%%%%%%%%%%%%%%%%%%%%%%%%%%%%%%%%%%%%%%%%%
\appendix       %%% starting appendix
\section*{Appendix A: Head of First Appendix}
Avoid Appendices if possible.

%%%%%%%%%%%%%%%%%%%%%%%%%%%%%%%%%%%%%%%%%%%%%%%%%%%%%%%%%%%%%%%%%%%%%%
\section*{Appendix B: Head of Second Appendix}
\subsection*{Subsection head in appendix}
The equation counter is not reset in an appendix and the numbers will
follow one continual sequence from the beginning of the article to the very end as shown in the following example.
\begin{equation}
a = b + c.
\end{equation}

\end{document}
