This section presents the background and purpose to the research. The research question is formulated in the end.

\subsection{Backgroud}
The background is stated in two perspectives. Since the target of this research is to explore the possibility of increasing the efficiency of one specific mechatronic design method by introducing new algorithm to the solver, those two aspects are concerning the design method and algorithm respectively.

\subsubsection{Integrated Design and Optimization of Mechatronic Products}
The development process of mechatronic products is challenging because it involves multiple engineering domains at the same time. In order to deal with them in an integrated manner as early as possible, several generic methodologies have been created, such as \textit{V-model} and \textit{Ulrich and Eppinger's model}. \textit{V-model}, presented in [1], features frequent verification and validation against the requirements during the integration phases. \textit{Ulrich and Eppinger's model} can be adopted in accordance with an unique context [2].

Although these methods have different formats, they do share something in common. According to [1] and [2], the detailed design is carried out based on rather abstract design in early phase, which means if some problems occur, chances are that the design in early stages needs to be modified as well. Moreover the lack of knowledge in early phase makes this kind of traceback inevitable [3]. Therefore the \textit{Integrated Design and Optimization of Mechatronic Products} (\textit{IDIOM}) is created to explore the potential of concept design with limited information.

The \textit{IDIOM} method treats the mechatronic system in a holistic pattern,  integrating different domains in early design phase and optimizing parameters of designed structure with respect to one or multiple targets. Fredrik Roos contributed to the models of the method [4] and Malmquist et al. extended the capability of the method to attack several optimal objects at the same time. At this stage, controller has been included in the process to achieve true synergistic integration.

There has been a model library set up inside the \textit{IDIOM} system with respect to both static and dynamic characteristics of those components. The approaches in different domains have been picked carefully to make the problem easy to solve under different configurations without too much lost detail. Therefore designers only need to pick and configure models from the model library and specify the design goal(s), the optimal solution of this specific design concept would be provided by the system [4]. 

\subsubsection{Generic Algorithm and Convex Optimization}

